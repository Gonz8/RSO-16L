\chapter{Spotkania zespołu}
\label{ap:2}

Poniżej znajduje się lista spotkań zespołu (z datami i ustaleniami):

\section{Spotkanie zespołu nr 1}

\textbf{Data spotkania:}  31 marca 2016

\textbf{Ustalenia:}

\begin{itemize}
\item Przedstawienie oczekiwań wobec ról w projekcie
\item Interpretacja i doprecyzowanie tematu projektu
\item Wybór danych medycznych jako danych przetwarzanych w ramach projektu
\item Przydział zadań na najbliższy czas:
	\begin{itemize}
	\item Dominik Giżyński - założenie repozytorium, instrukcja korzystania oraz dokument dobrych praktyk obowiązujących w projekcie
	\item Piotr Kuciński - doprecyzowanie wymagań przedmiotu, przegląd algorytmów do wykorzystania w projekcie
	\item Włodzimierz Szewczyk - doprecyzowanie wymagań przedmiotu, przegląd algorytmów do wykorzystania w projekcie
	\item Michał Herman - szkielet dokumentacji, wybór bazy danych używanej w projekcie
	\item Magda Malenda - architektura systemu, podział systemu na moduły, specyfikacja wymagań, wybór bazy danych używanej w projekcie
	\item Joanna Ohradka - koncepcja wykorzystania narzędzia Docker
	\item Tomasz Rydzewski - harmonogram projektu, specyfikacja wymagań
	\end{itemize}
\end{itemize}



\section{Spotkanie zespołu nr 2}

\textbf{Data spotkania:} 19 kwietnia 2016

\textbf{Ustalenia:}

\begin{itemize}
\item Porzucenie pomysłu klastrowej bazy danych
\item Definiowanie zadań na drugi etap
\item Spisanie koncepcji architektury, wymagań i zastosowania Dockera (do 23 kwietnia)
\item Zebranie i ujednolicenie dokumentacji (do 27 kwietnia)
\end{itemize}


\section{Spotkanie zespołu nr 3}

\textbf{Data spotkania:} 10 maja 2016

\textbf{Ustalenia:}

\begin{itemize}
\item Omówienie poprawek do etapu pierwszego
\item Cotygodniowe spotkania (termin wybrany później ankietą)
\item W przypadku potrzeby częstszych spotkań w mniejszym gronie - telekonferencje
\item Ustalenie rodzaju przetwarzania danych : statystyka i anonimizacja
\item Przeniesienie komunikacji w ramach zadań na GitHub
\item Przydział zadań na drugi etap:
	\begin{itemize}
	\item Dominik Giżyński - prototyp aplikacji
	\item Piotr Kuciński - plan testów
	\item Włodzimierz Szewczyk - uruchamianie / zatrzymywanie aplikacji, scenariusz końcowej demonstracji projektu
	\item Michał Herman - zebranie poprawionych dokumentacji pierwszego etapu, zredagowanie ustaleń ze spotkań zespołu, wyszukanie aktów prawnych dotyczących ochrony danych osobowych, wygenerowanie przykładowych danych do bazy
	\item Magda Malenda - szczegółowy opis rozwiązania, protokół komunikacyjny
	\item Joanna Ohradka - protokół spójności, szkolenia docker
	\end{itemize} 
\end{itemize}

\section{Spotkanie zespołu nr 4}

\textbf{Data spotkania:} 17 maja 2016

\textbf{Ustalenia:}

\begin{itemize}
\item Omówienie postępu prac
\item Harmonogram na następne dwa tygodnie (do końca drugiego etapu):
	\begin{itemize}
	\item Interfejs protokołu komunikacyjnego (do 20 maja) - osoby odpowiedzialne : Piotr Kuciński i Magda Malenda
	\item Poprawiona i uzupełniona dokumentacja oraz raport z postępu prac (do 26 maja) - Michał Herman
	\item Połączenie poszczególnych modułów (24 maja) - Dominik Giżyński i Włodzimierz Szewczyk
	\end{itemize}
\item Potrzeba środowiska wirtualnego (przygotowanie maszyny wirtualnej - Piotr Kuciński)
\end{itemize}


\section{Spotkanie zespołu nr 5}

\textbf{Data spotkania:} 17 maja 2016

\textbf{Ustalenia:}

\begin{itemize}
\item Dwie wersje klienta (jedna pobierająca - komunikuje się z serwerami warstwy zewnętrznej, druga dodająca/usuwająca dane z bazy - komunikuje się bezpośrednio z serwerami warstwy wewnętrznej).
\item Przydział kolejnych zadań:
	\begin{itemize}
	\item Piotr Kuciński - klient
	\item Joanna Ohradka - protokół spójności
	\item Magda Malenda - węzeł warstwy wewnętrznej (bez statystyki danych)
	\item Włodzimierz Szewczyk - prezentacja końcowa
	\item Dominik Giżyński - węzeł warstwy zewnętrznej
	\item Michał Herman - zasilenie bazy danych przykładowymi danymi, statystyka
	\end{itemize}
\item Potrzeba środowiska wirtualnego (przygotowanie maszyny wirtualnej - Piotr Kuciński)
\end{itemize}